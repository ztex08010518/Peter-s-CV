%%%%%%%%%%%%%%%%%
% This is an sample CV template created using altacv.cls
% (v1.1.4, 27 July 2018) written by LianTze Lim (liantze@gmail.com). Now compiles with pdfLaTeX, XeLaTeX and LuaLaTeX.
% 
%% It may be distributed and/or modified under the
%% conditions of the LaTeX Project Public License, either version 1.3
%% of this license or (at your option) any later version.
%% The latest version of this license is in
%%    http://www.latex-project.org/lppl.txt
%% and version 1.3 or later is part of all distributions of LaTeX
%% version 2003/12/01 or later.
%%%%%%%%%%%%%%%%

%% If you need to pass whatever options to xcolor
\PassOptionsToPackage{dvipsnames}{xcolor}

%% If you are using \orcid or academicons
%% icons, make sure you have the academicons 
%% option here, and compile with XeLaTeX
%% or LuaLaTeX.
% \documentclass[10pt,a4paper,academicons]{altacv}

%% Use the "normalphoto" option if you want a normal photo instead of cropped to a circle
% \documentclass[10pt,a4paper,normalphoto]{altacv}

\documentclass[10pt,a4paper]{altacv}

%% AltaCV uses the fontawesome and academicon fonts
%% and packages. 
%% See texdoc.net/pkg/fontawecome and http://texdoc.net/pkg/academicons for full list of symbols.
%% 
%% Compile with LuaLaTeX for best results. If you
%% want to use XeLaTeX, you may need to install
%% Academicons.ttf in your operating system's font 
%% folder.


% Change the page layout if you need to
\geometry{left=1cm,right=9cm,marginparwidth=6.8cm,marginparsep=1.2cm,top=1.25cm,bottom=1.25cm,footskip=2\baselineskip}

% Change the font if you want to.

% If using pdflatex:
\usepackage[T1]{fontenc}
\usepackage[utf8]{inputenc}
\usepackage[default]{lato}
\usepackage[UTF8]{ctex}


% If using xelatex or lualatex:
% \setmainfont{Lato}

% Change the colours if you want to
\definecolor{Mulberry}{HTML}{72243D}
\definecolor{SlateGrey}{HTML}{2E2E2E}
\definecolor{LightGrey}{HTML}{666666}
\colorlet{heading}{Sepia}
\colorlet{accent}{Mulberry}
\colorlet{emphasis}{SlateGrey}
\colorlet{body}{LightGrey}

% Change the bullets for itemize and rating marker
% for \cvskill if you want to
\renewcommand{\itemmarker}{{\small\textbullet}}
\renewcommand{\ratingmarker}{\faCircle}
%% sample.bib contains your publications
\addbibresource{sample.bib}

\usepackage[colorlinks]{hyperref}

\begin{document}

\name{江沛澤}
\tagline{}
\photo{2.8cm}{forcv}
\personalinfo{%
  % Not all of these are required!
  % You can add your own with \printinfo{symbol}{detail}
  \email{ztex080104518@gmail.com }
  \phone{0988-822605}
  \location{台中市 西區}
  \github{github.com/ztex080104518}
  %% You MUST add the academicons option to \documentclass, then compile with LuaLaTeX or XeLaTeX, if you want to use \orcid or other academicons commands.
%   \orcid{orcid.org/0000-0000-0000-0000}
}

%% Make the header extend all the way to the right, if you want. 
\begin{fullwidth}
\makecvheader
\end{fullwidth}

%% Depending on your tastes, you may want to make fonts of itemize environments slightly smaller
% \AtBeginEnvironment{itemize}{\small}


%% Provide the file name containing the sidebar contents as an optional parameter to \cvsection.
%% You can always just use \marginpar{...} if you do
%% not need to align the top of the contents to any
%% \cvsection title in the "main" bar.
\cvsection[page1sidebar]{動機}
\begin{itemize}
\item 我畢業於臺灣師範大學資工系,並於大四期間到瑞典Uppsala大學Information technology交流一年,,在那段時間體驗了不同的大學生活,印象最深刻的是教授授課的方式,幾乎每堂課都有期末專題,但不限制主題,只需運用課堂上教過的理論或演算法即可,與來自各國的學生一起集思廣益,瑞典的學生很有想法而且會去找外面的公司合作,課堂結束後也持續的研究,把研究當成是一種興趣,coding只是實現我們想法的工具,\textbf{讓興趣變成你的專業,才能在這條路上走得更深更遠}。
\item 畢業後,輾轉到一家公司,他們設備的資料庫建得很完善,可是缺乏資訊人員整理,這讓我有機會運用大學所學,幫工廠實現了部分的無紙化與資訊化,但面對龐大的資料,我知道還有其他更有效的方法可以讓流程更優化與簡化,甚者可以建立預測模組,這方面我還有許多不足的地方,\textbf{希望研究所可以往資料探勘或是AI的領域深究}。
\end{itemize}

\cvsection{Projects}
\cvevent{Machine learning}{Evolution of Fractal Flames for Image Similarity}{}{}
\begin{itemize}
\item We implement genetic algorithm for breeding a class of artificial images called Fractal Flames, with the goal of reproducing a given target image.
\item Combined with Local Binary Patterns (LBP) and Histograms of Oriented Gradients (HOG) as methods for comparing the similarity of two images.
\end{itemize}
\divider

\cvevent{科技部專題}{個人化適性程式學習輔助平台}{}{}
\begin{itemize}
\item 設計出一個程式學習的網站,可以根據使用者的程度來推薦適合他們的程式練習題,加入一些回饋機制,增加他們的學習動機
\item 架構 Ruby on Rails
\item 這個專案我參與前半段,投稿到網站設計與架設,後半段完善與資料搜集部分因exchange program所以由我的隊友們完成
\end{itemize}

\clearpage
\section{專業科目成績單}

\begin{center}
\begin{tabular}{|c|c|c|c|c|c|}
\hline
學年     & 學期 & 課程名稱       & 學分    & 成績 & 必/選 \\ \hline
102    & 1  & 程式設計(一)    & 3     & 78 & 必   \\ \hline
102    & 1  & 計算機概論      & 3     & 91 & 必   \\ \hline
102    & 1  & 微積分乙(一)    & 3     & 72 & 必   \\ \hline
102    & 2  & 程式設計(二)    & 3     & 83 & 必   \\ \hline
102    & 2  & 離散數學       & 3     & 96 & 必   \\ \hline
102    & 2  & 微積分乙(二)    & 3     & 84 & 必   \\ \hline
103    & 1  & 資料結構       & 3     & 82 & 必   \\ \hline
103    & 1  & 機率論        & 3     & 85 & 必   \\ \hline
103    & 1  & Linear Algebra       & 3     & 88 & 必   \\ \hline
103    & 2  & 演算法        & 3     & 86 & 必   \\ \hline
103    & 2  & 計算機結構      & 3     & 79 & 必   \\ \hline
102    & 1  & 基礎電子學      & 3     & 85 & 選   \\ \hline
102    & 1  & 基礎電子學實驗    & 1     & 90 & 選   \\ \hline
103    & 1  & 組合語言       & 3     & 68 & 選   \\ \hline
103    & 1  & 邏輯語言程式設計   & 3     & 86 & 選   \\ \hline
103    & 2  & 計算機網路      & 3     & 78 & 選   \\ \hline
103    & 2  & 程式語言結構     & 3     & 77 & 選   \\ \hline
103    & 2  & 人工智慧       & 3     & 76 & 選   \\ \hline
104    & 1  & 資訊安全       & 3     & A+ & 選   \\ \hline
104    & 1  & 計算機圖學      & 3     & A+ & 選   \\ \hline
104    & 1  & 系統程式       & 3     & A- & 選   \\ \hline
104    & 1  & 軟體工程       & 3     & A  & 選   \\ \hline
104    & 1  & 進階程式設計     & 3     & A- & 選   \\ \hline
104    & 1  & 自動機理論與正規語言 & 3     & A- & 選   \\ \hline
專業科目平均 &    &            & 80.1分 &    &     \\ \hline
\end{tabular}
\end{center}
\end{document}
